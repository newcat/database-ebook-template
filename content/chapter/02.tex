\chapter{Introduction to Document Stores}

A document store or document-oriented database system, is a storage concept characterized by its schema-free organization of data [1]. 
A document store can contain several collections, each of which contains one or more records/documents as shown in Figure 1 [2].

Unlike relational database systems, document stores allow a non-uniform structure of their records [1]. This means that each record can have individual columns [1]. Furthermore the data types of the different columns can be different from record to record [1]. 
Contrary to relational databases, columns can have multiple values and data records can have a nested structure as shown in Figure 1 [1].

Figure 2: Example of two document store records [5]
Most document stores use notations, which can be easily processed by applications accessing the stored data [1]. The most common data notation is JavaScript Object Notation (JSON) [1] as it supports a key-value serialization of the data without predefining the type of the value [3].
Important examples of document stores are MongoDB, Amazon DynamoDB, Couchbase and CouchDB [1]. Elasticsearch primarily is a search engine, which uses a distributed document store as its secondary database model [4].

