\chapter{Implementation and Experience with the Database System}

\section{Query Syntax}
Elasticsearch provides a comprehensive REST API to interact with the cluster \autocite{elastic2019_08}. This REST API can be used to manage the cluster itself, as well as to perform CRUD (Create, Read, Update, and Delete) operations and more complex search operations \autocite{elastic2019_08}. To define the queries Elasticsearch offers a full Query DSL (Domain Specific Language) based on JSON \autocite{elastic2019_08}.

In the following, the syntax of the queries is shown using the basic CRUD operations as examples:


\begin{minipage}[c]{0.95\textwidth}
    \textbf{Creating an index}
    (corresponds to a table in a relational databases)
    \begin{lstlisting}
    PUT /test
    {
        "settings": {
            "number_of_shards": 1
        },
        "mappings": {
            "_doc": {
                "properties": {
                    "name": { "type": "text" }
                }
            }
        }
    }
    \end{lstlisting}
\end{minipage}

\begin{minipage}[c]{0.95\textwidth}
    \textbf{Inserting a document}
    \begin{lstlisting}
    PUT /test/_doc/1
    {
        "name": "Max"
    }
    \end{lstlisting}
\end{minipage}

\begin{minipage}[c]{0.95\textwidth}
    \textbf{Reading the document}
    \begin{lstlisting}
    GET /test/_doc/1
    \end{lstlisting}
\end{minipage}

\begin{minipage}[c]{0.95\textwidth}
    \textbf{Updating the document}
    \begin{lstlisting}
    POST /test/_doc/1/_update
    {
        "doc": { "name": "Moritz" }
    }
    \end{lstlisting}
\end{minipage}

\begin{minipage}[c]{0.95\textwidth}
    \textbf{Deleting the document}
    \begin{lstlisting}
    DELETE /test/_doc/1
    \end{lstlisting}
\end{minipage}

The examples above illustrate how the REST API is using the different HTTP methods but the used JSON is rather simple. The following gives a example, showing a more complex search query:

\begin{minipage}[c]{0.95\textwidth}
    \begin{lstlisting}
    GET /person/_search
    {
        "query": { "match": { "sex": "male" } },
        "sort": [{ "name": "desc" }],
        "_source": ["name", "age"]
    }
    \end{lstlisting}
\end{minipage}

This query searches all documents in an index called “person”. It returns the fields “name” and “age” of all documents where “sex” equals “male” ordered descending by the field “name”.