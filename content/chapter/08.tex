\chapter{Advantages and Disadvanteges}

The main focus of elasticsearch is the (full-text) search and it is strongly optimized to fulfill this use case. Thus it offers the user a fully functional system, consisting of many already implemented concepts and algorithms, meaning that the developer does not have to do this him/herself.

A cornerstone and one of the reasons why the Elasticsearch project has been launched is to create a distributed system that is highly scalable. This allows the use in large enterprise use cases.

Another advantage of Elasticsearch is that it provides the ability to use the elastic stack to implement a large number of business use cases (see 5.1 The Elastic (former: ELK) Stack). It can also be used with Hadoop.


On the other hand, there are also some difficulties or disadvantages that come with using elasticsearch, like for example the obligation to buy a non-free license if one wants to use the security features.

Elasticsearch is a large and complex system, which is reflected in its configuration. With elasticsearch it is possible to start highly optimized (full-text) searches, but elasticsearch has to be configured accordingly, which can be very complex and time consuming. 

Another obstacle is the poor update performance of elasticsearch. As mentioned before, every document in elasticsearch is immutable. This means that in order to change an existing document, the old version must first be deleted and the new version inserted as a new document. Deleting and recreating is of course slower than changing an existing document as done in other document based databases.

An additional disadvantage is that elasticsearch is only eventual consistent in its architecture. If a new documentation is inserted and a search is started immediately, it is possible that the new document will not be displayed. This can lead to problems depending on the use case.
