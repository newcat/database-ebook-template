\chapter{History and General Information}

Elasticsearch is an open source search and analytics engine written in java, with its main focus on being distributed. It falls under the category of document-oriented databases and builds on the Apache Lucene library. Parts of the software are not licensed under open source licenses, but under commercial licenses, such as all security features.

The strengths of elasticsearch lie in the (full-text) search in a large amount of documents that are indexed when inserted using Apache Lucene. Elasticsearch offers its functions through an HTTP API using json documents.

Elasticsearch is the most popular search engine (as of April 2019) and is used in many enterprise use cases, as part of the Elastic Stack (see 5.1 The Elastic (former: ELK) Stack).

The project was started by Shay Banon as successor of Compass. Shay Banon wanted to dedicate himself to the task of developing a scalable search engine. He decided against a third version of Compass because it would be necessary to rewrite large parts of the software. So he started the project Elasticsearch to create "a solution built from the ground up to be distributed". The first version was released in February 2010 and has been continuously developed ever since.
